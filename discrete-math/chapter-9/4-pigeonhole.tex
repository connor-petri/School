\subsection{The Pigeonhole Principle}
\hrulefill
\paragraph*{Definition}
The \textbf{pigeonhole principle} states that if $n$ pigeons fly into $m$ pigeonholes, and if $n > m$, then at least one pigeonhole must contain more than one pigeon. 
In other words, a function from one finite set to a smaller finite set cannot be injective (one-to-one).

\paragraph*{Generalization}
For any function $f$ from a finite set $X$ with $n$ elements to a finite set $Y$ with $m$ elements and for any 
positive integer $k$, if $k < \frac{n}{m}$, then there exists at least one element in $Y$ such that y is the image
of at least $k + 1$ elements of $X$.

\paragraph*{Example}
Consider a set of 10 socks and 9 drawers. If we place each sock in a drawer, at least one drawer must contain at least two socks, since there are more socks than drawers.
\paragraph*{Applications}
The pigeonhole principle is widely used in combinatorics, computer science, and various fields of mathematics. It can be applied to prove the existence of certain properties in sets, such as:
\begin{itemize}
    \item In any group of 13 people, at least two people must have the same birthday (assuming 12 months).
    \item In a set of 52 playing cards, at least two cards must have the same suit.
    \item In a classroom with 30 students, at least two students must have the same shoe size if there are only 29 different sizes available.
    \item In a set of 100 integers, at least two integers must have the same remainder when divided by 10 (since there are only 10 possible remainders: 0 through 9).
\end{itemize}