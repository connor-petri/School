\subsection{Possibility Trees and Multiplication-Rule}
\hrulefill

\paragraph*{Definition}
A \textbf{possibility tree} is a graphical representation of all possible outcomes of a random process, where each branch represents a choice or event leading to further choices or outcomes.

\subsubsection*{The Multiplication Rule}
The \textbf{multiplication rule} states that if a process can be broken down into $k$ stages, where the first stage 
has $n_1$ possible outcomes, the second stage has $n_2$ possible outcomes, and so on up to the $k$-th stage with 
$n_k$ possible outcomes, then the total number of possible outcomes for the entire process is given by:
\begin{equation*}
    N = n_1 \times n_2 \times \ldots \times n_k
\end{equation*}

\subsubsection*{Permutation}
A \textbf{permutation} of a set is an arrangement of its elements in a specific order. The number of permutations of $n$ distinct objects is given by:
\begin{equation*}
    P(n) = n!
\end{equation*}
where $n!$ (n factorial) is the product of all positive integers up to $n$.

\paragraph*{Example} You can arrange the letters int the word "COMPUTER" in $8!$ different ways, since there are 8 distinct letters.

\subsubsection*{r-Permutation}
An \textbf{r-permutation} of a set of $n$ distinct objects is an arrangement of $r$ objects selected from the set. The number of r-permutations is given by:
\begin{equation*}
    P(n, r) = \frac{n!}{(n - r)!}
\end{equation*}
where $n$ is the total number of objects and $r$ is the number of objects to arrange.

\paragraph*{Example} The number of ways to arrange 3 letters from the word "COMPUTER" is given by:
\begin{equation*}
    P(8, 3) = \frac{8!}{(8 - 3)!} = \frac{8!}{5!} = 8 \times 7 \times 6 = 336
\end{equation*}