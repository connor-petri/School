\subsection{Counting Elements of Disjoint Sets}
\hrulefill
\subsubsection*{Addition Rule}
The \textbf{addition rule} states that if $A$ and $B$ are two disjoint sets (i.e., they have no elements in common), then the number of elements in the union of these sets is given by:
\begin{equation*}
    |A \cup B| = |A| + |B|
\end{equation*}

\subsubsection*{Difference Rule}
The \textbf{difference rule} states that if $A$ and $B$ are two sets, then the number of elements in the difference of these sets is given by:
\begin{equation*}
    |A - B| = |A| - |B \cap A|
\end{equation*}
If S is a finite sample space and A is an event in S, then:
\begin{equation*}
    P(A^c) = 1 - P(A)
\end{equation*}

\paragraph*{Example} Consider tossing a coin 5 times.
\begin{itemize}
    \item There are $6$ possible outcomes where there is exactly one head.
    \item There are $2^5 = 1 = 31$ possible outcomes where there is at least one head.
\end{itemize}

\subsubsection*{The includion/exclusion rule for Two or Three Sets}
If $A$, $B$, and $C$ are finite sets, then:
\begin{align*}
    N(A \cup B) &= |A| + |B| - |A \cap B| \\
    N(A \cup B \cup C) &= |A| + |B| + |C| - |A \cap B| - |A \cap C| - |B \cap C| + |A \cap B \cap C|
\end{align*}

