\subsection{Mathematical Induction 1: Proving Formulas}
\hrulefill

\subsubsection*{Method of Proof by Induction}
\paragraph*{Definition}
Induction proof explores the \textbf{patterns} we recognize from a list of unknown terms.

\paragraph*{Method}
Consider the statement $\forall n \in \{a \in \mathbb{Z} : n \geq a\}, P(n)$
\begin{itemize}
    \item Step 1: \textbf{(basis step):} Show that P(a) is true.
    \item Step 2: \textbf{(inductive step):} Show that if we suppose P(k) is true, then P(k+1) is true.
\end{itemize}

\paragraph*{Example}
Use the formula to evaluate $1 + 2 + \cdots + n = \frac{n(n+1)}{2}$.
\begin{align*}
    & \text{Suppose } n = 50\\
    1 + 2 + \cdots + 50 &= \frac{50(50+1)}{2}\\
    &= \frac{50(51)}{2}\\
    &= \frac{2550}{2}\\
    &= 1275
\end{align*}


\paragraph*{Definition} If a sum with a variable number of terms is show to equal an expression that does not contain either an ellipsis 
or a summation sign, we can say that the sum is written in \textbf{closed form}.

\paragraph*{Example}
Use the formula to evaluate $1 + 2 + \cdots + n$

\subsubsection*{Geometric Series}
\paragraph*{Definition}
If $r \in \mathbb{R} \land r \neq 1$, the sum of the first n terms of a geometric series is given by:
\begin{equation*}
    \sum_{i=0}^{n} r^i = \frac{r^{n+1} - 1}{r - 1}
\end{equation*}

\paragraph*{Example}
Use the above formula to evaluate $1 + 3 + \cdots + 3^{m-2}$
\begin{align*}
    r = 3, n = m-2\\
    1 + 3 + \cdots + 3^{m-2} &= \sum_{i=0}^{m-2} 3^i\\
    &= \frac{3^{m-1} - 1}{3 - 1} = \frac{3^{m-1}-1}{2}\\
\end{align*}

\paragraph*{Example}
$3^2 + 3^3 + \cdots + 3^{m}$
\begin{align*}
   $3^2 + 3^3 + \cdots + 3^{m}$ = 1 + 3 + 3^2 + 3^3 + \cdots + 3^{m} - (1 + 3)\\
   r=3, n=m\\
\end{align*}