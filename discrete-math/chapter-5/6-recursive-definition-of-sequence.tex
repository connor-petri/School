\subsection{Defining Sequences Recursively}
\hrulefill\\
\subsubsection*{Definition}
A sequence is defined recursively if it is defined in terms of itself.
\begin{itemize}
    \item The \textbf{initial conditions} are given for $a_0, a_1, \dots, a_{i-1}$ for a fixed $i$.
    \item A \textbf{recurrence relation} is a formula that relates $a_k$, $k \geq i$ to its predecessors $a_{k-1}, a_{k-2}, \dots, a_{0}$.
\end{itemize}

Two sequences are \textbf{equivalent} if they have the same initial conditions and recurrence relations that yeild the same sequence.

\paragraph*{Example}
\begin{equation*}
    a_k = 
    \begin{cases}
        1, & k = 1\\
        3, & k = 2\\
        2a_{k-1} + a_{k-2}, & \forall k \in \{x \in \mathbb{Z} : x \geq 3\}
    \end{cases}
\end{equation*}

\subsubsection*{The Fibonacci Sequence}
\paragraph*{Definition}
The Fibonacci sequence is a recursive sequence that is found everywhere in the natural world. It is defined as:
\begin{equation*}
    F_k =
    \begin{cases}
        1, & k = 0\\
        1, & k = 1\\
        F_{k-1} + F_{k-2}, & \forall k \in \{x \in \mathbb{Z} : x \geq 2\}
    \end{cases}
\end{equation*}