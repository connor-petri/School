\subsection{Divisibility}
\hrulefill

\paragraph*{Definitions}
If n and d are integers and d $\neq$ 0, then n is divisible by d if and only if n = dk for some integer k.
\begin{itemize}
    \item Notation: $d|n$ is read "d divides n".
    \begin{itemize}
        \item $d|n \iff \exists k \in \mathbb{Z} \ni n=dk$
    \end{itemize}
\end{itemize}

It is equivalent to the following statements:
\begin{itemize}
    \item n is a multiple of d
    \item d is a factor of n
    \item d is a divisor of n
    \item d divides n
\end{itemize}

\paragraph*{Example}
Prove the following statement: $\forall a,b,c \in \mathbb{Z}, a|b \land a|c \implies a|(b+c)$.
\begin{align*}
    & \text{Suppose } a,b,c \in \mathbb{Z} \land a|b \land a|c\\
    \therefore \quad & b=ak, \exists k \in \mathbb{Z} & \text{Definition of Divisibility}\\
    \therefore \quad & c=am, \exists m \in \mathbb{Z}\\
    \therefore \quad & b+c=a(k+m) & \text{Substitution and distributive}\\
    \therefore \quad & k+m \in \mathbb{Z} & \text{Integers are closed under addition}\\
    \therefore \quad & a|(b+c) & \text{Def. of divisibility}
\end{align*}

\subsubsection*{Divisibility Theorems}
\paragraph*{Positive Divisor of a Positive Integer Theorem}
\begin{equation*}
    \forall a,b \in \mathbb{Z}, a>0 \land b>0 \land a|b \implies a \leq b.
\end{equation*}

\paragraph*{Divisors of 1 Theroem}
The only divisors of 1 are 1 and -1.

\paragraph*{Transistivity of Divisibility Theorem}
\begin{equation*}
    \forall a,b,c \in \mathbb{Z}, a|b \land b|c \implies a|c
\end{equation*}

\paragraph*{Divisible by a Prime Theorem}
Any integers n > 1 is divisible by a prime number.

\paragraph*{Unique Factorization of Integers Theorem}
Given any integers n > 1, there exists k many distinct prime numbers $(p_1, \dots, p_k)$ and k many positive integers 
$(e_1, \dots, e_k)$, where k is a positive integer, such that:
\begin{equation*}
    n = \prod_{i=1}^k p_i^{e_i}
\end{equation*}

\paragraph*{Example}
If $a = \prod_{i=1}^k p_i^{e_i}$, find the standard factored form of $a^2$:
\begin{align*}
    a^2 &= \prod_{i=1}^k p_i^{e_i} \cdot \prod_{i=1}^k p_i^{e_i}\\
    &= (p_1^{e_1}p_2^{e_2} \cdots p_k^{e_k}) \cdot (p_1^{e_1}p_2^{e_2} \cdots p_k^{e_k})\\
    &= p_1^{2e_1}p_2^{2e_2} \cdots p_k^{2e_k}\\
    &= \prod_{i=1}^k p_i^{2e_i}
\end{align*}

