\subsection{The Quotient-Remainder Theorem}

\paragraph*{Theorem}
\begin{equation*}
    \forall n \in \mathbb{Z}, \forall d \in \mathbb{Z}^+, \exists q,r \in \mathbb{Z} \ni n=dq+r \land 0 \leq r < d
\end{equation*}

\paragraph*{Definition}
Given any integer n and any positive integer d:
\begin{align*}
    n \div d = q\\
    n \% d = r
\end{align*}

\paragraph*{Example}
If today is tuesday, what day of the week will it be in 365 days?
\begin{equation*}
    365 \mod 7 = 1
    \text{Tuesday + 1 day} = \text{Wednesday}
\end{equation*}

\subsubsection*{The Parity Property}
\paragraph*{Definition}
We call the fact that any integer is either even or odd the parity property.
\paragraph*(Method of Proof by Division Into Cases)
To prove a statement of the form "If $A_1 or A_2 \dots or A_n$, then C."

\paragraph*{Example}
The product of two consecutive integers is even.
\begin{align*}
    & \exists n \in \mathbb{Z}\\
    \\
    & \text{Case 1: } n \text{ is even}\\
    \therefore \quad & n = 2k, \exists k \in \mathbb{Z}\\
    \therefore \quad & n + 1 = 2k + 1\\
    \therefore \quad & n(n+1) = 2k(2k+1) = \\
    \therefore \quad & n(n+1) = 2(2k^2 + k)\\
    \therefore \quad & n(n+1) \text{ is even}\\
    \\
    & \text{Case 2: } n \text{ is odd}\\
    \therefore \quad & n = 2k + 1, \exists k \in \mathbb{Z}\\
    \therefore \quad & n + 1 = 2k + 2\\
    \therefore \quad & n(n+1) = (2k + 1)(2k + 2)
    \therefore \quad & n(n+1) = 2((k + 1)(2k + 1))\\
    \therefore \quad & n(n+1) \text{ is even}
\end{align*}

