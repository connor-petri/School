\subsection{The Quotient-Remainder Theorem}

\paragraph*{Theorem}
\begin{equation*}
    \forall n \in \mathbb{Z}, \forall d \in \mathbb{Z}^+, \exists q,r \in \mathbb{Z} \ni n=dq+r \land 0 \leq r < d
\end{equation*}

\paragraph*{Definition}
Given any integer n and any positive integer d:
\begin{align*}
    n \div d = q\\
    n \mod d = r
\end{align*}

\paragraph*{Example}
If today is tuesday, what day of the week will it be in 365 days?
\begin{align*}
    365 \mod 7 = 1
    \text{Tuesday + 1 day} = \text{Wednesday}
\end{align*}

\subsubsection*{The Parity Property}
\paragraph*{Definition}
We call the fact that any integer is either even or odd the parity property.
\paragraph*(Method of Proof by Division Into Cases)
To prove a statement of the form "If $A_1 or A_2 \dots or A_n$, then C."

\paragraph*{Example}
The product of two consecutive integers is even.
\begin{align*}
    & \exists n \in \mathbb{Z}\\
    \\
    & \text{Case 1: } 2|n\\
    \therefore \quad & 2|n \implies \exists k \in \mathbb{Z} \ni n = 2k \implies n + 1 = 2k + 1\\
    \therefore \quad & n(n+1) = 2k(2k+1) = 2(2k^2 + k)\\
    \therefore \quad & k \in \mathbb{Z} \implies 2k^2+k \in \mathbb{Z}\\
    \therefore \quad & n(n+1) = 2(2k^2 + k) \land 2k^2 + k \in \mathbb{Z} \implies [2|n(n+1)]\\
    \\
    & \text{Case 2: } \neg (2|n)\\
    \therefore \quad & \neg (2|n) \implies \exists k \in \mathbb{Z} \ni n = 2k + 1 \implies n + 1 = 2k + 2\\
    \therefore \quad & n(n+1) = (2k + 1)(2k + 2) = 2(2k^2 + 3k + 1)\\
    \therefore \quad & k \in \mathbb{Z} \implies 2k^2 + 3k + 1 \in \mathbb{Z}\\
    \therefore \quad & n(n+1) = 2(2k^2 + 3k + 1) \land 2k^2 + 3k + 1 \in \mathbb{Z} \implies [2|n(n+1)]
    \\
    \therefore \quad & [2|n(n+1)]
\end{align*}

\subsubsection*{Absolute Value}
\paragraph*{Definition}
For any real number x, the absolute value of x, delotes |x|, is defined as:
\begin{equation*}
    |x| = \begin{cases}
        x & \text{if } x \geq 0\\
        -x & \text{if } x < 0
    \end{cases}
\end{equation*}

\paragraph*{Lemma}
\begin{align*}
    & \forall r \in \mathbb{R}, -|r| \leq r \leq |r|\\
    & \forall r \in \mathbb{R}, |-r| = |r|\\
\end{align*}

\paragraph*{The Triangle Inequality}
\begin{equation*}
    \forall x,y \in \mathbb{R}, |x+y| \leq |x| + |y|
\end{equation*}