\subsection{RC Circuits}
\hrulefill

\paragraph*{Definition}
An RC circuit is a circuit that contains a resistor and a capacitor. The capacitor stores energy in the form of an electric field, 
while the resistor dissipates energy in the form of heat. The time constant $\tau$ of an RC circuit is given by the product of the 
total resistance and capacitance in the circuit.

\begin{equation*}
    \tau = RC
\end{equation*}

Where $\tau$ is the time constant in seconds, $R$ is the resistance in ohms, and $C$ is the capacitance in farads. 

\paragraph*{Charging a Capacitor}
First, we must update our max charge equation to account for emf.

\begin{equation*}
    Q_{max} = C\varepsilon
\end{equation*}

Where $Q_{max}$ is the max charge the capacitor can hold in Coulombs, C is the capacitance of the capacitor in Farads, and $\varepsilon$
is the electromotive force in volts.\\

We can then find the charge and current of the capacitor as a function of time.

\begin{align*}
    q(t) \quad &= \quad C\varepsilon(1-e^{-t/\tau}) \quad = \quad Q_{max}(1-e^{-t/\tau})\\
    I(t) \quad &= \quad \frac{dq}{dt} \quad = \quad \frac{\varepsilon}{R}e^{-t/\tau} \quad = \quad \frac{q(t) - Q_{max}}{\tau}\\
\end{align*}

Where $q(t)$ is the charge on the capacitor at time $t$ in Coulombs, $I(t)$ is the current through the capacitor at time $t$ in Amps,
$C$ is the capacitance of the capacitor in Farads, $\varepsilon$ is the electromotive force in volts, $R$ is the resistance in ohms,
and $\tau$ is the time constant of the circuit in seconds.

\paragraph*{Discharging a Capacitor}
We can also find the charge and current of a capacitor as it discharges as a function of time.

\begin{align*}
    q(t) &= Q_ie^{-t/\tau}\\
    I(t) &= -\frac{Q_i}{\tau}e^{-t/\tau}\\
\end{align*}

Where $Q_i$ is the initial charge on the capacitor in Coulombs, and the rest of teh variables having the same definition as above.