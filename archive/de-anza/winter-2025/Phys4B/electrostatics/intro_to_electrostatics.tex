\subsection{Introduction to Electrostatics}

\hrulefill

\subsubsection*{Charge}

\paragraph*{Definition}
Much like inertia, charge is a fundamental property of matter that describes how an object interacts with 
electric fields. It is measured in Coulombs $[C]$. Practically, charge indicates if the object has an excess 
or deficiency of electrons.

\paragraph*{Quantization of Charge}
Charge is quantized, meaning that it can only exist in discrete values. 
The smallest possible charge is the charge of an electron. Protons and electrons carry equal but opposite charges, 
represented by $e$.

\begin{equation*}
    q = \pm Ne \Longrightarrow e = 1.6 \times 10^{-19}C
\end{equation*}

Where N is any integer. This means that the charge of an object is always a multiple of the charge of an electron.

\paragraph*{Conservation of Charge}
Much like energy and matter, charge is conserved in a closed system. This means that the total charge in a system
will remain constant. 

\begin{equation*}
    \sum q_i = \sum q_f
\end{equation*}


\paragraph*{Conductors and Insulators}
Electrical conductors are materials in which some of the electrons are not bound to atoms and can move relatively freely
through the material. This allows for the easy transfer of charge. Metals are the most common conductors. 
Electrical insulators are materials in which all of the electrons are bound to atoms and cannot move freely. This impededs
the transfer of charge. Glass, rubber, and plastic are common insulators.\\



\subsubsection*{Coloumb's Law}

\paragraph*{Definition}
Coulomb's law describes the fundimental force between 2 charged objects. It is given by the equation:

\begin{equation*}
    F_e = k_e \frac{q_1q_2}{r^2} \Longrightarrow \vec{F}_{12} = k_e \frac{q_1q_2}{r^2}\hat{r}_{12}
\end{equation*}

Where $F_e$ is the electrostatic force in Newtons, $k_e$ is Coulomb's constant, $q_1$ and $q_2$ are the charges of the objects in Coulombs,
$r$ is the distance between the objects in meters, and $\hat{r}_{12}$ is a unit vector pointing from object 1 to object 2.\\

Notice the similarities between Coulomb's law and Newton's law of gravitation. These similarities are because both are fundamental forces of nature.
Both are inverse square laws, meaning that the force between the objects decreases exponentially as the distance between them increases. Both forces are
proportional to a property of matter (mass for gravity and charge for electrostatics) and a constant ($G$ and $k_e$ respectively).\\

The main difference is that electrostatic forces can be attractive or repulsive, while gravity is always attractive. This is because charge can be positive
or negative, while it is (for our purposes) impossible to have negative mass.\\

