\subsection{The Stack}
\paragraph*{Definition}
The stack is a region of memory that is used to store data temporarily. It is a LIFO (Last In, First Out) data structure. The stack is used to 
store local variables, function arguments, and return addresses. The stack pointer register \verb|esp| points to the top of the stack. The stack 
grows downward in memory.

\paragraph*{Instructions}
\begin{description}
    \item[push \{Mem/Reg/Literal\}] Pushes a value onto the stack
    \item[pop \{Mem/Reg/Literal\}] Pops a value from the stack
\end{description}

\hrulefill

\subsubsection*{Passing arguements using the stack}
\paragraph*{Setting up stack frame}
The stack frame register \verb|ebp| is used to point to the base of the current stack frame. This is used to access local variables and function arguments
explicitly.
\begin{lstlisting}
foo proc

    push ebp        ; save the old base pointer
    mov ebp, esp    ; set the base pointer to the 
                    ; current stack pointer

    mov eax, [ebp+8] ; access the first argument
    mov ebx, [ebp+12] ; access the second argument

    pop ebp         ; restore the old base pointer
    ret 12          ; return and clean up the stack (12 bytes)
                    ; (8 bytes for the arguments, 
                    ; 4 bytes for return address)

foo endp
\end{lstlisting}

